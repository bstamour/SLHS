\documentclass[a4paper]{article}

\title{A Monadic Framework for Constructing Uncertain Reasoning Systems}
\author{Bryan St. Amour}

\begin{document}

\maketitle

\section{The problem addressed}

\par
The main problem that we wish to address is that aside from the Java demos provided
by Josang, there does not appear to be any off-the-shelf framework for experimenting
with, and embedding subjective logic into applications. Software that requires a form
of artifical reasoning and wish to utilize subjective logic appear to be doomed to
rewrite the core operators by hand, which can lead to buggy, slow software.

\section{Our approach}

\par
We propose to develop a framework for subjective logic that

\begin{itemize}
	\item Can be utilized as a standalone workbench, much like a Read-eval-print loop such as various lisps.
	\item Can be embedded into larger applications which require artificial reasoning.
	\item Is easily extendable to support more kinds of probablistic logics (e.g. Dempster-shafer theory).
	\item Is efficient.
\end{itemize}

\par
We address the above items in the proceeding sections.
Our language of implementation is to be the pure functional programming language Haskell, as it's purity
and non-strict semantics offer various advantages over contemporary imperative languages.

\subsection{A Standalone Workbench}

\par
We plan to leverage the GHC compiler's read-eval-print loop in order to create an easy-to-use environment
for which users can model and explore situations that require uncertain reasoning. Assuming a UNIX-like
environment, a typical session might look like the following:

\begin{itemize}
	\item $> \mbox{ghci}$
	\item $> \mbox{import SL}$
	\item $> \mbox{data Atoms} = \mbox{Red} | \mbox{Green} | \mbox{Blue}$
	\item $> \mbox{let mass = fromList} [ ([Red], 1/4), ([Green], 1/4), ([Green, Blue], 1/3) ]$
	\item $> \mbox{expectation} [Blue]$
	\item $\mbox{Some value here}$
\end{itemize}

\par
The workbench will allow the users to experiment with the full range of Subjective Logic operators in
an incremental manner.


\subsection{An embeddable reasoning engine}

\par
Our framework will comprise of a set of modules that can be imported not only from GHC's REPL, as seen
in the previous section, but also from arbitrary Haskell programs. This allows users to construct
real-world applications that utilize our framework. 

\par
Furthermore we expect to construct a foreign interface to the C programming language. This will allow
for users to interface with our framework from any language that supports calling arbitrary C functions,
including (but not exhaustive)

\begin{itemize}
	\item C++
	\item Python
	\item Ruby
\end{itemize}




\subsection{Ease of extension}

\par
We plan to utilize monads from Category Theory as the functional abstraction
to support such generality.

\subsection{Efficiency}

\par
The naive descriptions of the operators in Subjective Logic have exponential runtime with respect
to the size of the frame of discernment because the most general form of opinion - the hyper
opinion - is computed over the powerset of the frame. We propose exploring the use of memoization
in order to curtain this runtime by re-using pre-calculated subexpressions.


\section{Justification for our approach}


\section{Expected outcomes}



\end{document}
