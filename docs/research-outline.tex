\documentclass[a4paper]{article}

\title{A Monadic Framework to Facilitate the Construction of Uncertain Reasoning Systems}
\author{Bryan St. Amour}

\begin{document}

\maketitle

\section{The problem addressed}

\par
The main problem that we wish to address is that aside from the Java demos provided
by Josang, there does not appear to be any off-the-shelf framework for experimenting
with, and embedding subjective logic into applications. Software that requires a form
of artifical reasoning and wish to utilize subjective logic appear to be doomed to
rewrite the core operators by hand, which can lead to buggy, slow software.

\section{Our approach}

\par
We propose to develop a framework for subjective logic that

\begin{itemize}
	\item Can be utilized as a standalone workbench, much like a Read-eval-print loop such as various lisps.
	\item Can be embedded into larger applications which require artificial reasoning.
	\item Is easily generalizeable to support more kinds of probablistic logics (e.g. Dempster-shafer theory).
	\item Is efficient.
\end{itemize}

\par
On item 3 above: we plan to utilize monads from Category Theory as the functional abstraction
to support such generality. As for efficiency, many of the core operators in Subjective Logic
are naively exponential in terms of the size of the frame of discernment. We plan on investigating
kinds of memoization and dynamic programming in order to curtail some of the computations.

\par
Our implementation will be done using the Haskell programming language. Haskell is a pure functional
language with non-strict semantics as the default evaluation method. Furthermore we will provide a
binding to the C programming langauge, which will allow our framework to be utilized by programs
written in C, C++, Python, etc. Any language that can call C subroutines will be able to take
advantage of our proposed framework.

\section{Justification for our approach}


\section{Expected outcomes}



\end{document}
