\documentclass[a4paper]{article}

\title{A Monadic Framework to Facilitate the Construction of Uncertain Reasoning Systems}
\author{Bryan St. Amour}

\begin{document}

\maketitle

\section{The problem addressed}

% Kind of ugly - reword this.
\par
The problem we pose to address in the thesis research is that there exist various
systems for reasoning with uncertainty. Two popular systems are the
Dempster-Shafer theory, and Subjective Logic. For researchers interested in modelling
situations with such systems, it can be burdensome to compare outcomes for the same
situations across different systems because, as far as we currently know, there do
not exist any workbenches designed to support multiple reasoning systems at the same
time.

\section{Our approach}

% TODO: Code snippets go in here.

\par
The approach we have taken is to develop an artificial reasoning platform that is
parameterized not only on the frame of discernment, but also the underlying
reasoning system. We are developing the platform using the Haskell programming
language, which allows us to leverage the strong type system in order to aide us
in determining the correctness of our platform.

\par
Our system relies extensively on monads from Category Theory to represent computations
within the reasoning platform. Monads have been used in computation for some time now,
and they are somewhat of a design pattern for pure functional programming.

\section{Justification for our approach}

\section{Expected outcomes}



\end{document}
