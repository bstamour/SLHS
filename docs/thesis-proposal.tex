\documentclass{beamer}

\usepackage{fancyvrb}
\DefineVerbatimEnvironment{code}{Verbatim}{fontsize=\small}
\DefineVerbatimEnvironment{example}{Verbatim}{fontsize=\small}
\newcommand{\ignore}[1]{}

\mode<presentation>
{ \usetheme{Antibes} }

\title{A Type-safe Framework for Uncertain Reasoning with Subjective Logic}

\author{Bryan St. Amour}

%\AtBeginSection[]
%{
%  \begin{frame}<beamer>
%  \frametitle{Outline}
%  \tableofcontents[currentsection]
%  \end{frame}
%}

\begin{document}

\begin{frame}
\titlepage
\end{frame}

\begin{frame}

Thesis committee

\begin{itemize}
  \item \emph{Advisor:} Dr. Robert Kent
  \item \emph{Internal Reader:} Dr. Ziad Kobti
  \item \emph{External Reader:} Dr. Rick Caron (Math/Stats)
\end{itemize}

\end{frame}

\begin{frame}
\tableofcontents
\end{frame}

%-------------------------------------------------------------------------------

\section{Introduction}

\begin{frame}
\frametitle{Introduction - Decision Support}

Decision support systems are systems that support decision making
through the use of analytics, modelling, etc \cite{sprague_framework_1980}.

Example application areas include

\begin{itemize}
  \item Clinical \cite{berner2007clinical}
  \item Business \cite{klein_knowledge-based}
  \item Health and Safety \cite{kent2010application}
\end{itemize}

\end{frame}

%-------------------------------------------------------------------------------

\section{Subjective Logic}

\begin{frame}
\frametitle{Subjective Logic}

Subjective Logic is a probablistic logic for uncertain reasoning
\cite{josang_logic_2001}.

Basic object is the subjective opinion, defined over a frame of discernment.

\begin{itemize}
  \item Binomial opinions
  \item Multinomial opinions
  \item Hyper opinions \cite{josang2012interpretation}
\end{itemize}

\end{frame}

\begin{frame}
\frametitle{Subj-logic - Frame of Discernment}

A set of atomic entities

$$\Theta = \lbrace Red, Blue, Green, Yellow \rbrace$$

Effectively, the universe of discourse.

We define opinions over subsets of $\Theta$.

\end{frame}

\begin{frame}
\frametitle{Subj-logic - binomial opinions}

Given a binary frame $\Theta = \lbrace x, \lnot x \rbrace$, we can define a
\emph{binomial opinion} of $x$.

$$\omega_{x}^A = \langle b_x, d_x, u_x, a_x \rangle$$

satisfying $b_x + d_x + u_x = 1$, and $0 \leq a_x \leq 1$.

\end{frame}

\begin{frame}
\frametitle{Subj-logic - binomial opinions}

For example...

$$\Theta = \lbrace Red, \lnot Red \rbrace$$

$$\omega_{Red}^{Bryan} = \langle 0.5, 0.4, 0.1, 0.5 \rangle$$

... maybe I'm colour-blind?

\end{frame}

\begin{frame}
\frametitle{Subj-logic - multinomial opinions}

For larger frames ($|\Theta| > 2$) we define \emph{multinomial opinions}.

$$ \omega_\Theta^A = \langle \vec{b}, u, \vec{a} \rangle $$

such that

$$u + \sum_{x \in \Theta} \vec{b}(x) = 1$$

and

$$\sum_{x\in \Theta} \vec{a}(x) = 1$$

\end{frame}

\begin{frame}
\frametitle{Subj-logic - hyper opinions}

Most general of all subjective opinions. Assigns belief mass to subsets of the
frame (elements of the reduced powerset.)

Reduced powerset: $R^\Theta = 2^\Theta \backslash \lbrace \emptyset, \Theta \rbrace$

\end{frame}

\begin{frame}
\frametitle{Subj-logic - hyper opinions}

e.g.

$$\Theta = \lbrace Red, Blue, Yellow \rbrace$$

$$\omega = \langle \vec{b}, u, \vec{a} \rangle$$
$$\vec{b}\left(\lbrace Red, Blue \rbrace \right) = \frac{1}{2}$$
$$\vec{b}\left(\lbrace Blue, Yellow \rbrace \right) = \frac{1}{4}$$
$$\vec{b}\left(\lbrace Yellow \rbrace \right) = \frac{1}{8}$$

\end{frame}

\begin{frame}
\frametitle{Subj-logic - hyper opinions}

Similar to multinomial opinions, rules apply

$$u + \sum_{x \in R^\Theta} \vec{b}\left(x\right) = 1$$

$$\sum_{x \in R^\Theta} \vec{a}\left(x\right) = 1$$

\end{frame}

\begin{frame}
\frametitle{Subj-logic - benefts}

\begin{itemize}
  \item Wealth of operators for working with beliefs
     \footnote{\url{http://folk.uio.no/josang/papers/subjective_logic.pdf}}
  \item Beta, Dirichlet, Hyper-dirichlet interpretations
  \item Easy to model many situations
    \cite{josang2008conditional}, \cite{josang2006trust}, \cite{kent2010application}
\end{itemize}

\end{frame}

\begin{frame}
\frametitle{Subj-logic - drawbacks}

\begin{itemize}
  \item No widely available frameworks/toolkits/libraries!
\end{itemize}

There exists an online demo\footnote{\url{http://folk.uio.no/josang/sl/BV.html}},
but we cannot find any other published implementations.

\end{frame}

%-------------------------------------------------------------------------------

\section{Research Focus}

\begin{frame}
\frametitle{A Subjective Logic Framework}

Can we develop a Subjective Logic framework that is

\begin{itemize}
  \item Type-safe
  \item Embeddable as a library
  \item Utilisable as a standalone workbench
\end{itemize}

\end{frame}

\begin{frame}
\frametitle{Similar Languages/Frameworks}

\begin{itemize}
  \item R (statistical language)
  \item SciPy/NumPy (libraries for Python)
\end{itemize}

\pause

Like R, should be usable by mathematicians/statisticians. Not just programmers.

\pause

Like SciPy/NumPy, retains full access to a general purpose programming language.

\pause

We propose to use Haskell :-)

\end{frame}

\begin{frame}
\frametitle{Haskell - overview}

\begin{itemize}
  \item Pure functional programming language
  \item General purpose
  \item Strong, statically typed
  \item Lazy evaluation
\end{itemize}

\end{frame}

\begin{frame}[fragile]
\frametitle{Example}

\begin{code}
  data Person = Alice | Bob
  data Color  = Red | Blue | Yellow
  beliefMass  = mass
    [ (Alice, [ (Item Red,  1/2)
              , (Item Blue, 1/3)
              , (Theta,     1/6)
              ])
    , (Bob,   [ (Item Red,          1/4)
              , (Items [Red, Blue], 1/4)
              , (Theta,             1/2)
              ])
    ]
\end{code}

\end{frame}

\begin{frame}[fragile]
\frametitle{Example cont'd}

\begin{code}
  baseRate = [ (Red,    1/3)
             , (Blue,   1/3)
             , (Yellow, 1/3)
             ]
\end{code}

\end{frame}

\begin{frame}[fragile]
\frametitle{Example cont'd}

\begin{code}
  opinion1 = binomialOpinion Alice Red
  opinion2 = binomialOpinion Alice Blue
  opinion3 = binomialOpinion Bob   Red
  opinion4 = binomialOpinion Bob   Blue

  expr     = belief $
               (opinion1 `or` opinion2) `concensus`
               (opinion2 `or` opinion3)

  result   = run expr beliefMass baseRate
\end{code}

\end{frame}

\begin{frame}
\frametitle{Benefits}

\begin{itemize}
  \item Type-safe with respect to opinions
  \item Type-safe with respect to belief holders
  \item Statements cannot be run outside of their context
\end{itemize}

\end{frame}

%-------------------------------------------------------------------------------

\section{Progress Thus Far}

%-------------------------------------------------------------------------------

\section{Roadmap Toward Completion}

%-------------------------------------------------------------------------------

\section{Conclusion}

%-------------------------------------------------------------------------------

\section{References}

\begin{frame}[allowframebreaks]
\frametitle{References}

\bibliographystyle{plain}
\bibliography{bibliography}

\end{frame}

%-------------------------------------------------------------------------------

\end{document}
