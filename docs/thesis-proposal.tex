\documentclass{beamer}

\usepackage{fancyvrb}
\usepackage{stmaryrd}

\DefineVerbatimEnvironment{code}{Verbatim}{fontsize=\small}
\DefineVerbatimEnvironment{example}{Verbatim}{fontsize=\small}
\newcommand{\ignore}[1]{}

\mode<presentation>
{ \usetheme{Antibes} }

\title{A Denotational Semantics for Subjective Logic}

\author{Bryan St. Amour}

%\AtBeginSection[]
%{
%  \begin{frame}<beamer>
%  \frametitle{Outline}
%  \tableofcontents[currentsection]
%  \end{frame}
%}

\begin{document}

%-------------------------------------------------------------------------------

% Front matter.

\begin{frame}
\titlepage
\end{frame}

\begin{frame}

Thesis committee

\begin{itemize}
  \item \emph{Advisor:} Dr. Robert Kent
  \item \emph{Internal Reader:} Dr. Ziad Kobti
  \item \emph{External Reader:} Dr. Rick Caron (Math/Stats)
\end{itemize}

\end{frame}

\begin{frame}
\tableofcontents
\end{frame}

%-------------------------------------------------------------------------------

% Thesis intro.

\section{Introduction}

\begin{frame}
\frametitle{Introduction - Decision Support}

Decision support systems are systems that support decision making
through the use of analytics, modelling, etc \cite{sprague_framework_1980}.

Example application areas include

\begin{itemize}
  \item Clinical \cite{berner2007clinical}
  \item Business \cite{klein_knowledge-based}
  \item Health and Safety \cite{kent2010application}
\end{itemize}

\end{frame}

%-------------------------------------------------------------------------------

% Introduction to Subjective Logic

\section{Subjective Logic}

\begin{frame}
\frametitle{Subjective Logic}

Subjective Logic is a probablistic logic for uncertain reasoning
\cite{josang_logic_2001}.

Basic object is the subjective opinion, defined over a frame of discernment.

\begin{itemize}
  \item Binomial opinions
  \item Multinomial opinions
  \item Hyper opinions \cite{josang2012interpretation}
\end{itemize}

\end{frame}

\begin{frame}
\frametitle{Subj-logic - Frame of Discernment}

A set of atomic entities

$$\Theta = \lbrace Red, Blue, Green, Yellow \rbrace$$

Effectively, the universe of discourse.

We define opinions over subsets of $\Theta$.

\end{frame}

\begin{frame}
\frametitle{Subj-logic - binomial opinions}

Given a binary frame $\Theta = \lbrace x, \lnot x \rbrace$, we can define a
\emph{binomial opinion} of $x$.

$$\omega_{x}^A = \langle b_x, d_x, u_x, a_x \rangle$$

satisfying $b_x + d_x + u_x = 1$, and $0 \leq a_x \leq 1$.

\end{frame}

\begin{frame}
\frametitle{Subj-logic - binomial opinions}

For example...

$$\Theta = \lbrace Red, \lnot Red \rbrace$$

$$\omega_{Red}^{Bryan} = \langle 0.5, 0.4, 0.1, 0.5 \rangle$$

... maybe I'm colour-blind?

\end{frame}

\begin{frame}
\frametitle{Subj-logic - multinomial opinions}

For larger frames ($|\Theta| > 2$) we define \emph{multinomial opinions}.

$$ \omega_\Theta^A = \langle \vec{b}, u, \vec{a} \rangle $$

such that

$$u + \sum_{x \in \Theta} \vec{b}(x) = 1$$

and

$$\sum_{x\in \Theta} \vec{a}(x) = 1$$

\end{frame}

\begin{frame}
\frametitle{Subj-logic - hyper opinions}

Most general of all subjective opinions. Assigns belief mass to subsets of the
frame (elements of the reduced powerset.)

Reduced powerset: $R^\Theta = 2^\Theta \backslash \lbrace \emptyset, \Theta \rbrace$

\end{frame}

\begin{frame}
\frametitle{Subj-logic - hyper opinions}

e.g.

$$\Theta = \lbrace Red, Blue, Yellow \rbrace$$

$$\omega = \langle \vec{b}, u, \vec{a} \rangle$$
$$\vec{b}\left(\lbrace Red, Blue \rbrace \right) = \frac{1}{2}$$
$$\vec{b}\left(\lbrace Blue, Yellow \rbrace \right) = \frac{1}{4}$$
$$\vec{b}\left(\lbrace Yellow \rbrace \right) = \frac{1}{8}$$

\end{frame}

\begin{frame}
\frametitle{Subj-logic - hyper opinions}

Similar to multinomial opinions, rules apply

$$u + \sum_{x \in R^\Theta} \vec{b}\left(x\right) = 1$$

$$\sum_{x \in R^\Theta} \vec{a}\left(x\right) = 1$$

\end{frame}

\begin{frame}
\frametitle{Subj-logic - benefts}

\begin{itemize}
  \item Wealth of operators for working with beliefs
     \footnote{\url{http://folk.uio.no/josang/papers/subjective_logic.pdf}}
  \item Beta, Dirichlet, Hyper-dirichlet interpretations
  \item Easy to model many situations
    \cite{josang2008conditional}, \cite{josang2006trust}, \cite{kent2010application}
\end{itemize}

\end{frame}

\begin{frame}
\frametitle{Subj-logic - drawbacks}

\begin{itemize}
  \item No widely available frameworks/toolkits/libraries!
\end{itemize}

There exists an online demo\footnote{\url{http://folk.uio.no/josang/sl/BV.html}},
but we cannot find any other published implementations.

\end{frame}

%-------------------------------------------------------------------------------

% Research focus + thesis statement.

\section{Thesis Statement}

\begin{frame}
\frametitle{Focus of research}

To construct a type-safe Subjective Logic implementation, we need to know the following:

\begin{itemize}
  \item What is the meaning of a Subjective Logic expression?
  \item How we can tell the valid expressions from the invalid ones?
\end{itemize}

Therefore, we need a formal semantics for Subjective Logic.

\end{frame}

\begin{frame}
\frametitle{Thesis statement}

If we can construct a formal semantics for Subjective Logic, then we can construct
type-safe implementations (frameworks, libraries, tools, etc.)

We propose to construct a \emph{denotational semantics} for Subjective Logic and use it as
a basis for a type-safe implementation in the \emph{Haskell} programming language.

\end{frame}

%-------------------------------------------------------------------------------

% Give a brief introduction to denotational semantics here.

\section{Denotational Semantics}

\begin{frame}
\frametitle{Denotational Semantics Overview}

\begin{itemize}
  \item Formal semantics
  \item popular for programming languages.
  \item Maps statements onto mathematical objects (e.g. lambda expressions)
  \item Compositional: meaning of statement is comprised of meaning of sub-statements.
\end{itemize}

$$\llbracket 4 + 5 \rrbracket \equiv \llbracket + \rrbracket \left(\llbracket 4 \rrbracket, \llbracket 5 \rrbracket\right)$$

\end{frame}

\begin{frame}
\frametitle{Denotational Semantics Overview}

An example from Subjective Logic...

\begin{equation}
\begin{split}
\llbracket \omega^{A}_{\mbox{x}} \land \omega^{A}_{\mbox{y}} \rrbracket
    & \equiv \llbracket \land \rrbracket \left(
                 \llbracket \omega^{A}_{\mbox{x}} \rrbracket,
                 \llbracket \omega^{A}_{\mbox{y}} \rrbracket
              \right) \\
    & \equiv \left(\lambda x. \lambda y. \mbox{logical-and(x,y)}\right)
             \llbracket \omega^{A}_{\mbox{x}} \rrbracket
             \llbracket \omega^{A}_{\mbox{y}} \rrbracket \\
    & \equiv \left(\lambda x. \lambda y. \mbox{logical-and(x,y)}\right)
             \mbox{binl-op(A, x)} \mbox{bin-op(A, y)} \\
    & \equiv \mbox{logical-and(bin-op(A, x), bin-op(A, y))}
\end{split}
\end{equation}

\end{frame}

\begin{frame}
\frametitle{A Denotational Semantics for Subjective Logic}

We propose to map subjective expressions to expressions in a \emph{typed lambda calculus}
(e.g. simply-typed $\lambda$-calculus, System-F, etc.)

$$
\llbracket \omega_{\mbox{x}} \rrbracket
    \equiv \lambda m : Mass. \mbox{opinion}\left(\llbracket x\rrbracket, m\right): \mbox{Op}
$$

$$
\llbracket \land \rrbracket
    \equiv \lambda x:\mbox{Op}.
             \lambda y:\mbox{Op}.
               \lambda m:\mbox{Mass}.
                 \mbox{logical-and}(\mbox{x m}, \mbox{y m}) : \mbox{Op}
$$

$$
\llbracket \alpha \odot \beta \rrbracket
    \equiv \llbracket \odot \rrbracket \left(\llbracket \alpha \rrbracket, \llbracket \beta \rrbracket\right)
$$

\end{frame}

\begin{frame}
\frametitle{A Denotational Semantics for Subjective Logic}

Types play an important role in Subjective Logic.

\begin{itemize}
  \item Frames of discernment must be the same (or not!)
  \item Belief holders must occasionally be the same (e.g. concensus)
  \item Cannot combine binomial with multinomial opinions directly
  \item Multinomials can only be constructed from \emph{dirichlet mass assignments}
  \item etc...
\end{itemize}

Expressions are valid \emph{iff} the types match.

\end{frame}

\begin{frame}
\frametitle{A Denotational Semantics for Subjective Logic}

By representing subjective expressions as typed $\lambda$-calculus terms, we can exclude
expressions that are invalid (i.e. they fail to \emph{type-check}.)

With a strongly typed language such as \emph{Haskell}, we can build systems that reject
\emph{absurd} expressions before they are ever compiled.

\end{frame}

%-------------------------------------------------------------------------------

% How far along are we toward completion of the work?

\section{Current Progress}


%-------------------------------------------------------------------------------

\section{Roadmap Toward Completion}

%-------------------------------------------------------------------------------

\section{Conclusion}

%-------------------------------------------------------------------------------

\section{References}

\begin{frame}[allowframebreaks]
\frametitle{References}

\bibliographystyle{plain}
\bibliography{bibliography}

\end{frame}

%-------------------------------------------------------------------------------

\end{document}
