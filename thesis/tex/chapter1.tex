\documentclass[thesis.tex]{subfiles}

\begin{document}

\chapter{Introduction}



\section{Reasoning Under Uncertainty}

Imagine two sensors designed to measure two very different properties of some
small, furry animals - say guinea pigs. One sensor measures the sex of the baby guinea
pig, while the other attempts (poorly) to detect the specific breed of the guinea pig.
For simplicity, assume that guinea pigs fall into one of the breeds specified in table
\ref{tbl:guinea-pig-breeds}. Assume guinea pigs have an even probability of being born
male or female. Once the data has been collected, the guinea pig farm manager wishes to compute the
liklihood of each sex/breed pairing. However the manager must take into account the
uncertainty corresponding to mixed-breed guiea pigs.

\begin{table}
  \begin{center}
    \begin{tabular}{| l | l |}
      \hline
      Name of Breed & \% of Total \\
      \hline
      Silky         & 25\%        \\
      American      & 40\%        \\
      Peruvian      & 25\%        \\
      Mixed         & 10\%        \\
      \hline
    \end{tabular}
  \end{center}

  \caption{Imaginary distribution of guinea pig breeds}
  \label{tbl:guinea-pig-breeds}
\end{table}

Now imagine a family doctor, investigating the symptoms of one of her patients. The patient
shows symptoms that could point to a myriad of actual causes. The doctor begins with three
potential hypotheses that could explain the symptops, and must abduce the correct explanation.

Lastly, suppose you and two of your friends with to see a movie. There are three movies currently
playing in your local theatre: Star Wars - The Empire Strikes First (M1), Cassablanca 2 (M2), and
A Slug's Life (M3). Each of you has a preference for each of the three movies, as depicted in
table \ref{tbl:movies}. Is it possible for the three of you to come to a reasonable decision for
which movie to see?

\begin{table}
  \begin{center}
    \begin{tabular}{ r|c|c|c| }
      \multicolumn{1}{r}{}
      &  \multicolumn{1}{c}{M1}
      &  \multicolumn{1}{c}{M2}
      &  \multicolumn{1}{c}{M3} \\
      \cline{2-4}
      You & 0.5 & 0.3 & 0.1 \\
      \cline{2-4}
      Bill & 0.2 & 0.6 & 0.2 \\
      \cline{2-4}
      Ted  & 0.7 & 0.0 & 0.0 \\
      \cline{2-4}
    \end{tabular}
  \end{center}

  \caption{Movie Preferences}
  \label{tbl:movies}
\end{table}

The above scenarios all share a common theme: they involve reasoning about uncertain or incomplete
data. Any system designed to aide end users in reasoning in many real-world scenarios must take
uncertainty into account. One particular form of probabilistic logic, Subjective Logic, offers a
variety of operators for many different scenarios. And because Subjective Logic has been
constructed to directly deal with the problem of uncertainty, it has been put to use in
many domains [cite some examples here].


\section{The Problem Addressed}

Subjective Logic is a relatively new form of probabilistic logic that is currently
under active development. The novelty of Subjective Logic is that it
directly handles uncertainty, and each and every operator for manipulating
Subjective opinions takes this uncertainty into account. The result is a
flexible calculus of opinions that can be used to model many kinds of
situations that require reasoning under uncertainty.

However, as Subjective Logic is still in its infancy, the operators, opinions, and
even in some cases nomenclature, are in a flux. The result of this is that
there is, to the best of our knowledge, no implementation of Subjective
Logic available for application developers. Furthermore, there does not
yet appear to be any formal analysis of the computational complexity of
the operators provided.


\section{Previous Work on This Problem}

Josang has provided an implementation of some Subjective Logic operators.
However the implementation is incomplete as it does not support computing
with general subjective opinions.

To the best of our knowledge, there does not yet exist any formal complexity
analysis of the operators of Subjective Logic.


\section{Our Proposed Solution}

In order to combat this scarcity, we propose to develop an implementation
of Subjective Logic using the Haskell programming language. Haskell is
a pure functional programming language that is lazily evaluated by default.
Haskell has gained popularity as an academic research language, but is also
seeing some uptake in industry. Alongside our implementation, we propose to
perform a complexity analysis on each and every operator provided.


\section{Organization of this Document}

The remainder of this document is organized as follows.
Chapter~\ref{chap:background-information}
introduces the reader to the relevant background information on uncertain reasoning,
Subjective Logic, and pure functional programming to hopefully allow the proceding
chapters to be understood. Chapter~\ref{chap:thesis-statement} contains the thesis
statement. Chapter~\ref{chap:sl-in-haskell} introduces \emph{SLHS}, a library of
Subjective Logic objects and operators, written in the Haskell programming language.
Chapter~\ref{chap:results-and-analysis} presents an analysis of complexity for each
operator in SLHS. It also contains examples of how one can use SLHS to model situations
that require uncertain reasoning. Chapter~\ref{chap:conclusion} concludes the document.




\end{document}
