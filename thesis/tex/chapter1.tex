\documentclass[thesis.tex]{subfiles}

\begin{document}

\chapter{Introduction}



\section{Motivation}

Imagine being in a courtroom where a man is being tried for murder. The prosecution has
brought forth three witnesses who allegedly observed the event. Witness A is a close friend
of the defendant, and has a high opinion of him. Witness B does not like the defendant at
all, and has a very negative opinion of him. The third witness, Witness C, has no prior
opinion of the defendant, and thus is very uncertain about his character.

The judge has never once interacted with the defendant and therefore must base his entire
opinion of him on the evidence brought forth and by the witness testimonies. The
judge does, however, have an opinion about each of the three witnesses. The judge golfs regularly with
witness A, the judge knows witness B is the pastor at a local church, and witness C is a
courtroom regular - always involved in some mischief or other. Therefore, while the judge
can construct an opinion of the defendant by analyzing the opinions of the three witnesses and forming a
consensus, he also takes into account his knowledge of the three, and discounts their
opinions by his own opinions of them. The judge places great weight on the testimonies of
witnesses A and B, and can barely belief a word of witness C's statement.

Now imagine two sensors designed to measure two orthogonal properties of baby guinea pigs.
Before they reach a certain age, male guinea pigs must be separated from their mothers
(and sisters) because they reach sexual maturity very quickly. Therefore it is important to
be able to measure the sex of the guinea pigs quickly and partition them
accordingly in order to avoid a combinatorial explosion of new children. Another important
measurable trait is the breed of the guinea pig. If the
pigs have been brought from many different litters, then it is important to be able to
classify them as \emph{Silky}, \emph{American}, or \emph{Peruvian} before sending them to the pet store.
This classification cannot be carried out with absolute certainty, as there can be mixed breed guinea pigs as well.
Assume for simplicity that guinea pigs have a male/female birth ratio of 50/50, and that the
probability of a guinea pig having a certain breed is given in Table \ref{tbl:guinea-pig-breeds}.

Given these two sensors, it is possible to classify the guinea pigs into eight categories.
To complicate matters, imagine that your breed-detecting sensor has a tendency to give back inaccurate
results, say, 5\% of the time. Any reasoning that is to be done with this sensor data must be
handled with care, as it has a non-zero rate of error.

\begin{table}
  \begin{center}
    \begin{tabular}{| l | l |}
      \hline
      Name of Breed & \% of Total \\
      \hline
      Silky         & 25\%        \\
      American      & 40\%        \\
      Peruvian      & 25\%        \\
      Mixed         & 10\%        \\
      \hline
    \end{tabular}
  \end{center}

  \caption{Imaginary distribution of guinea pig breeds}
  \label{tbl:guinea-pig-breeds}
\end{table}

Lastly, suppose you and two of your friends wish to see a movie. There are three movies currently
playing in your local theatre: Star Wars - The Empire Strikes First (M1), Casablanca 2 (M2), and
A Slug's Life (M3). Each of you has a preference for each of the three movies, as depicted in
Table \ref{tbl:movies}. Is it possible for the three of you to come to a reasonable decision for
which movie to see?

\begin{table}
  \begin{center}
    \begin{tabular}{ r|c|c|c| }
      \multicolumn{1}{r}{}
      &  \multicolumn{1}{c}{M1}
      &  \multicolumn{1}{c}{M2}
      &  \multicolumn{1}{c}{M3} \\
      \cline{2-4}
      You & 0.5 & 0.3 & 0.1 \\
      \cline{2-4}
      Bill & 0.2 & 0.6 & 0.2 \\
      \cline{2-4}
      Ted  & 0.7 & 0.0 & 0.0 \\
      \cline{2-4}
    \end{tabular}
  \end{center}

  \caption{Movie Preferences}
  \label{tbl:movies}
\end{table}

The above scenarios all share a common theme: they involve reasoning about uncertain or incomplete
data. Many real-world reasoning scenarios must deal with this kind of data, and thus any
automated system designed to aide decision-makers in these (and many other) kinds of situations must be able to take
uncertainty into account.

This thesis is about the engineering of a library for constructing and evaluating expressions in
\emph{Subjective Logic}, a recently emergent extension to probabilistic logic \cite{josang2001logic}
with support for reasoning under uncertainty.
The library is designed to be a central component of Unified Data Management
and Decision Support System (UDMDSS) \cite{kent2010towards,  kobti2011towards, kent2011design},
a decision support system that is under active research and development within our lab. We utilize the
\emph{Haskell} programming language \cite{hudak1992report} as it supports strong typing, has
excellent support for programming with \emph{monads} \cite{peyton1993imperative}, and is overall an
elegant \emph{purely functional} programming language for implementing mathematical programs.





\section{The Problem Addressed}

Subjective Logic is a relatively new form of probabilistic logic that is currently
under active development \cite{josang2001logic}. The novelty of Subjective Logic is that it
directly handles uncertainty, and each and every operator for manipulating
subjective opinions - the primary objects of Subjective Logic -
takes this uncertainty into account. The result is a
flexible calculus of opinions that can be used to model many kinds of
situations that require reasoning under uncertainty \cite{pope2005analysis, josang2006trust, li2004trust, oren2007subjective}.

As Subjective Logic is still an area of active research, the operators, opinions,
and even nomenclature, are evolving. As a result of this
there is, to the best of our knowledge, no implementation of Subjective
Logic available for use by application developers and researchers.
Audun Josang has provided an implementation of some Subjective Logic operators,
however the implementation is incomplete. The implementation was constructed before
Subjective Logic had introduced \emph{hyper opinions} and other operators now found
in the literature.





\section{Our Proposed Solution}

To combat this scarcity of implementations, we have developed a library of Subjective Logic
operators using the \emph{Haskell} programming language. We represent
expressions of Subjective Logic as functions from an initial world state to some
numeric output, and the operators of Subjective Logic as higher order functions.
Therefore simple expressions of Subjective Logic can be combined to form larger
more complex equations.

In order to assist us in combining together these equations, we use monads,
in particular a \emph{state monad}. Monads are ubiquitous in Haskell, and are
a general design pattern that has been previously used to represent stateful computations \cite{launchbury1994lazy},
input/output \cite{peyton1993imperative},
and formal \cite{hutton1998monadic, leijen2001parsec} and natural language
\cite{hafiz2010lazy} parsers.

In order to demonstrate the effectiveness of our library, we utilize it to implement some
example calculations provided by Josang in the literature. Furthermore we prove that our set of
operators terminates for all possible valid input equations.
Lastly, we perform a complexity analysis on a representative subset of the operators.

We expect that our library will be found useful by the research community, and that
it will spur the development of Subjective Logic-based reasoning applications.





\section{Thesis Contribution}

To realize the solution proposed above, in this thesis we have done the following:

\begin{itemize}
  \item We developed SLHS, a Subjective Logic library that is type-safe, efficient,
    and compositional, using the Haskell programming language (Chapter \ref{chap:sl-in-haskell}).
  \item We contributed two additional operators to Subjective Logic (Section \ref{sec:extensions-to-sl}).
  \item We proved that the evaluator of SLHS (the function that evaluates the Subjective Logic expressions)
    terminates for all valid Subjective Logic expressions (Section \ref{sec:termination}).
  \item We analyzed the time complexity of a representative subset of the Subjective Logic operators (Section \ref{sec:complexity}).
  \item We constructed example applications to demonstrate the effectiveness and ease of use of SLHS (Section \ref{sec:examples}).
\end{itemize}











\section{Organization of this Document}

The remainder of this document is organized as follows.
Chapter \ref{chap:background-information}
introduces the reader to the relevant background information on decision support systems,
automated reasoning systems, uncertain reasoning, Subjective Logic, and pure functional
programming in Haskell to allow the proceeding
chapters to be better understood. Chapter \ref{chap:thesis-statement} contains the thesis problem, hypothesis, objectives, and methodology.
Chapter \ref{chap:sl-in-haskell} introduces \emph{SLHS}, a library of
Subjective Logic objects and operators, written in the Haskell programming language.
Chapter \ref{chap:results-and-analysis} presents a proof of termination, analysis of complexity a sample of
operators in SLHS, and a discussion regarding the use of Haskell and monads on the design of the library.
It also contains examples of how one can use SLHS to model situations
that require uncertain reasoning, and lastly, it discusses the library's role within the larger UDMDSS decision support system.
Chapter \ref{chap:conclusion} concludes this thesis and discusses areas for future improvement.




\end{document}
