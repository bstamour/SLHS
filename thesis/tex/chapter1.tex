\documentclass[thesis.tex]{subfiles}

\begin{document}

\chapter{Introduction}



\section{Motivation}

Imagine being in a courtroom where a man is being tried for murder. The prosecution has
brought forth three witnesses who allegedly observed the event. Witness A is a close friend
of the defendant, and has a high opinion of him. Witness B does not like the defendent at
all, and has a very negative opinion of him. The third witness, Witness C, has no prior
opinion of the defendent, and thus is very uncertain about his character.

The judge has never once interacted with the defendent and therefore must base his entire
opinion of him on the evidence brought forth, and by the witness testimonies. However the
judge does have an opinion on each of the three witnesses. The judge golfs regularly with
witness A, the judge knows witness B as the pastor at a local church, and witness C is a
courtroom regular - always involved in some mischief or other. Therefore, while the judge
can construct an opinion by analyzing the opinions of the three witnesses and forming a
consensus, he also takes into account his knowledge of the three, and discounts their
opinions by his own opinions of them. The judge places great weight on the testimonies of
witnesses A and B, and can barely belief a word of witness C's statement.

Now imagine two sensors designed to measure two orthogonal properties of baby guinea pigs.
Before they reach a certain age, male guinea pigs must be separated from their mothers
(and sisters) because they reach sexual maturity very quickly. Therefore it is important to
be able to measure the sex of the guinea pigs quickly and partition the little furballs
accordingly. Another important measurable trait is the breed of the guinea pig. If the
pigs have been brought from many different litters, then it is important to be able to
classify them as \emph{Silky}, \emph{American}, or \emph{Peruvian}. However this classification
cannot be carried out with absolute certainty, as there can be mixed breed guinea pigs as well.
Assume for simplicity that guinea pigs have a male/female birth ratio of 50/50, and that the
probability of a guinea pig having a certain breed is given in table \ref{tbl:guinea-pig-breeds}.

Given these two sensors, it is possible to combine the two guinea pig traits in order to
classify them. However, imagine that your breed sensors has a tendency to give back inaccurate
results, say, 5\% of the time. Any reasoning that is to be done with this sensor data must be
discounted by the knowledge that it has a non-zero rate of error.

\begin{table}
  \begin{center}
    \begin{tabular}{| l | l |}
      \hline
      Name of Breed & \% of Total \\
      \hline
      Silky         & 25\%        \\
      American      & 40\%        \\
      Peruvian      & 25\%        \\
      Mixed         & 10\%        \\
      \hline
    \end{tabular}
  \end{center}

  \caption{Imaginary distribution of guinea pig breeds}
  \label{tbl:guinea-pig-breeds}
\end{table}

Lastly, suppose you and two of your friends with to see a movie. There are three movies currently
playing in your local theatre: Star Wars - The Empire Strikes First (M1), Cassablanca 2 (M2), and
A Slug's Life (M3). Each of you has a preference for each of the three movies, as depicted in
table \ref{tbl:movies}. Is it possible for the three of you to come to a reasonable decision for
which movie to see?

\begin{table}
  \begin{center}
    \begin{tabular}{ r|c|c|c| }
      \multicolumn{1}{r}{}
      &  \multicolumn{1}{c}{M1}
      &  \multicolumn{1}{c}{M2}
      &  \multicolumn{1}{c}{M3} \\
      \cline{2-4}
      You & 0.5 & 0.3 & 0.1 \\
      \cline{2-4}
      Bill & 0.2 & 0.6 & 0.2 \\
      \cline{2-4}
      Ted  & 0.7 & 0.0 & 0.0 \\
      \cline{2-4}
    \end{tabular}
  \end{center}

  \caption{Movie Preferences}
  \label{tbl:movies}
\end{table}

The above scenarios all share a common theme: they involve reasoning about uncertain or incomplete
data. Many real-world reasoning scenarios must deal with this kind of data, and thus any
automated system designed to aide decision-makers must be able to take into account uncertainty.
Subjective Logic, a recently emergent probabilistic logic system, offers a wealth of operators
for combining, discounting, and reasoning with, uncertain data.


\section{The Problem Addressed}

Subjective Logic is a relatively new form of probabilistic logic that is currently
under active development. The novelty of Subjective Logic is that it
directly handles uncertainty, and each and every operator for manipulating
Subjective opinions takes this uncertainty into account. The result is a
flexible calculus of opinions that can be used to model many kinds of
situations that require reasoning under uncertainty.

However, as Subjective Logic is still in its infancy, the operators, opinions, and
even in some cases nomenclature, are in a flux. The result of this is that
there is, to the best of our knowledge, no implementation of Subjective
Logic available for application developers. Furthermore, there does not
yet appear to be any formal analysis of the computational complexity of
the operators provided.

Josang has provided an implementation of some Subjective Logic operators.
However the implementation is incomplete as it does not support computing
with general subjective opinions.





\section{Our Proposed Solution}

To combat this scarcity, we propose to develop a library of Subjective Logic
operators using the \emph{Haskell} programming language. We propose to represent
expressions of Subjective Logic as functions from an initial world state to some
numeric output, and the operators of Subjective Logic as higher order functions.
Therefore simple expressions of Subjective Logic can be combined to form larger
more complex equations.

In order to assist us in combining together these equations, we will use monads,
in particular a \emph{state monad}. Monads are ubiquitous in Haskell, and are
general enough to represent stateful computations \cite{launchbury1994lazy},
input/output \cite{peyton1993imperative},
and formal \cite{hutton1998monadic, leijen2001parsec} and natural language
\cite{hafiz2010lazy} parsers.

In order to evaluate our library for correctness, we propose to test them against
example calculations provided by Josang in the literature. Furthermore we propose
to prove that our set of operators terminates for all possible input equations.
Lastly, we propose to perform a complexity analysis on a sample of the operators.

We hope that our library will be found useful by the research community, and that
it will spur the development of Subjective Logic-based reasoning applications.






\section{Organization of this Document}

The remainder of this document is organized as follows.
Chapter~\ref{chap:background-information}
introduces the reader to the relevant background information on uncertain reasoning,
Subjective Logic, and pure functional programming to hopefully allow the proceding
chapters to be understood. Chapter~\ref{chap:thesis-statement} contains the thesis
statement. Chapter~\ref{chap:sl-in-haskell} introduces \emph{SLHS}, a library of
Subjective Logic objects and operators, written in the Haskell programming language.
Chapter~\ref{chap:results-and-analysis} presents an analysis of complexity for each
operator in SLHS. It also contains examples of how one can use SLHS to model situations
that require uncertain reasoning. Chapter~\ref{chap:conclusion} concludes the document.




\end{document}
