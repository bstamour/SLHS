\documentclass[thesis.tex]{subfiles}

\begin{document}

\chapter{Introduction}


\section{The Problem Addressed}

Subjective Logic is a form of probabilistic logic that is currently
under active development. The novelty of Subjective Logic is that it
directly handles uncertainty, and each and every operator for manipulating
Subjective opinions takes this uncertainty into account. The result is a
flexible calculus of opinions that can be used to model many kinds of
situations that require reasoning under uncertainty.

As Subjective Logic is still in its infancy, the operators, opinions, and
even in some cases nomenclature, are in a flux. The result of this is that
there is, to the best of our knowledge, no implementation of Subjective
Logic available for application developers. Furthermore, there does not
yet appear to be any formal analysis of the computational complexity of
the operators provided.


\section{Previous Work on This Problem}

Josang has provided an implementation of some Subjective Logic operators.
However the implementation is incomplete as it does not support computing
with general subjective opinions.

To the best of our knowledge, there does not yet exist any formal complexity
analysis of the operators of Subjective Logic.


\section{Our Proposed Solution}

In order to combat this scarcity, we propose to develop an implementation
of Subjective Logic using the Haskell programming language. Haskell is
a pure functional programming language that is lazily evaluated by default.
Haskell has gained popularity as an academic research language, but is also
seeing some uptake in industry. Alongside our implementation, we propose to
perform a complexity analysis on each and every operator provided.


\section{Organization of this Document}

The remainder of this document is organized as follows.
Chapter~\ref{chap:background-information}
introduces the reader to the relevant background information on uncertain reasoning,
Subjective Logic, and pure functional programming to hopefully allow the proceding
chapters to be understood. Chapter~\ref{chap:thesis-statement} contains the thesis
statement. Chapter~\ref{chap:sl-in-haskell} introduces \emph{SLHS}, a library of
Subjective Logic objects and operators, written in the Haskell programming language.
Chapter~\ref{chap:results-and-analysis} presents an analysis of complexity for each
operator in SLHS. It also contains examples of how one can use SLHS to model situations
that require uncertain reasoning. Chapter~\ref{chap:conclusion} concludes the document.




\end{document}
