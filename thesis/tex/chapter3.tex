\documentclass[thesis.tex]{subfiles}

\begin{document}

\chapter{Thesis Statement}
\label{chap:thesis-statement}

In this chapter we describe the problem that this thesis addresses, our thesis hypothesis, and
our research objectives. Lastly we outline the methodology that we followed in order to achieve
those mentioned objectives.



\section{Thesis Problem}

%
% Reiterate the motivation, then state the problem.
%

As mentioned previously, there does not yet exist a comprehensive library of Subjective
Logic operators that can be used for research, development, and experimentation. There
exists a partial implementation of Subjective Logic operators by Audun Josang\footnote{\url{http://folk.uio.no/josang/sl/Op.html}},
but at the time of this writing, to our knowledge no complete implementation exists.

We expect that such a library of operators should be efficient, type-safe, and compositional.
The library should be efficient in such a way that values are only computed as needed. The
library should be type-safe in order to catch invalid Subjective Logic expressions as early as
possible. By leveraging a strong type system, the library should be able to catch many errors
at the time of compilation. Finally, the library should be compositional in a sense that
arbitrarily complicated Subjective Logic expressions should be able to be constructed from a
small set of functions and operators.


\section{Thesis Hypothesis}

Motivated by the aforementioned problem, our hypothesis for this thesis is: Using monads and strong typing, it is
possible to construct a general purpose Subjective Logic library that is type-safe, efficient,
and compositional.




%
% Mention the asthetic quality of Haskell's functional style: function composition vs
% writing an extension in Java or whatever the fuck in prolog.
%



\section{Objectives}

The objectives of our research are the following:

\begin{itemize}
  \item Develop a library of Subjective Logic operators using monadic higher order functions.
  \item Demonstrate the type safety of the library.
  \item Prove that the expression evaluator, the \emph{run} function, terminates for all valid Subjective Logic expressions.
  \item Analyze the time complexity of a representative subset of the operators.
\end{itemize}




\section{Methodology}

In order to satisfy the objectives of our research, we undertake the following:

\begin{itemize}
  \item We developed the library using the Haskell programming language due to it's strong type
system and excellent support for monadic programming.
  \item We discuss how Haskell's strong type system allows for our library to reject certain
classes of ill-formed Subjective Logic equations.
  \item We utilize structural induction on the length of the input equation to prove that
our operators terminate.
  \item We analyze the time complexity of the operators based on the amount of elements in the
frame of discernment that have non-zero belief mass assigned to them.
\end{itemize}

In the following chapter we will discuss the implementation of \emph{SLHS}: Subjective Logic in Haskell.
Then, in Chapter \ref{chap:results-and-analysis} we will provide proofs of termination, complexity
analysis, and discuss how Haskell's strong type system allows our library to reject a large class of ill-formed
Subjective Logic expressions.





\end{document}
