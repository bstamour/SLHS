\documentclass[thesis.tex]{subfiles}

\begin{document}

\chapter{Thesis Statement}
\label{chap:thesis-statement}

In this chapter we will describe our research motivations, and then present our thesis hypothesis and
research objectives. Lastly we will outline the methodology that we will utilize in order to
achieve our goals.


\section{Research Motivation}

As mentioned previously, there does not yet exist a comprehensive library of Subjective
Logic operators that can be used for application development and experimentation. There
exists a partial implementation by Josang [cite], but at the time of this writing, to our
knowledge no other implementation exists.

We expect that such a library of operators should be efficient, type-safe, and compositional.
The library should be efficient in such a way that values are only computed as needed. The
library should be type-safe in order to catch invalid Subjective Logic equations as early as
possible. By leveraging a strong type system, the library should be able to catch many errors
at the time of compilation. Finally, the library should be compositional in a sense that
arbitrarily complicated Subjective Logic equations should be able to be constructed from a
small set of functions and operators.


\section{Thesis Hypothesis}

Given our motivation, our hypothesis for this thesis is: Using monads and strong typing, it is
possible to construct a general purpose Subjective Logic library that is type-safe, efficient,
and compositional.




%
% Mention the asthetic quality of Haskell's functional style: function composition vs
% writing an extension in Java or whatever the fuck in prolog.
%



\section{Objectives}

The objectives of our research are the following:

\begin{itemize}
  \item Develop a monadic combinator library of Subjective Logic operators.
  \item Analyze the type safety of our library.
  \item Prove that our set of operators terminates for all Subjective Logic equations.
  \item Analyze the complexity of a selection of the operators.
\end{itemize}


\section{Methodology}

In order to complete the objectives of our research, we will do the following:

\begin{itemize}
  \item We will develop the library using the Haskell programming language due to it's strong type
system and excellent support for monadic programming.
  \item We will discuss how haskell's strong type system allows for our library to reject certain
classes of ill-formed Subjective Logic equations.
  \item We will utilize structural induction on the length of the input equation to prove that
our operators terminate.
  \item We will analyze the time complexity of the operators based on the amount of elements in the
frame of discernment that have non-zero belief mass assigned to them.
\end{itemize}

In the following chapter we will discuss the implementation of \emph{SLHS}: Subjective Logic in Haskell.
Then, in chapter \ref{chap:results-and-analysis} we will provide proofs of termination, complexity
analysis, and discuss how Haskell's strong type system allows our library to reject some ill-formed
Subjective Logic equations.





\end{document}
