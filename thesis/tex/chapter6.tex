\documentclass[thesis.tex]{subfiles}

\begin{document}

\chapter{Conclusion}
\label{chap:conclusion}

In this chapter we present our concluding remarks, as well as discuss possible avenues for
future improvements to the SLHS library.



\section{Conclusion}

For this thesis we constructed a Subjective Logic library, SLHS, that uses monadic higher
order functions to represent subjective expressions. Subjective Logic is a relatively new
extension to probabilistic logic \cite{josang2001logic} that directly handles uncertainty
in each and every operator. The fundamental unit for computation is the \emph{subjective opinion},
which is a combination of belief mass assigned to a frame of discernment, plus a scalar value
representing uncertainty.

Within this thesis, we have shown the construction of SLHS in Chapter \ref{chap:sl-in-haskell},
discussed its current limitations in Section \ref{sec:limitations}, shown its termination in
Section \ref{sec:termination}, and analyzed a representative subset of the operators in
Section \ref{sec:complexity}. Furthermore we have discussed the role that \emph{Haskell}, our
language of implementation, has had on SLHS in Section \ref{sec:types}, and how the use of
\emph{monads} simplified our code (Section \ref{sec:monads}). In totality, we have shown that it
is possible to construct a Subjective Logic library that is type-safe, efficient, and compositional.



\section{Future Work}

In this section we discuss areas for future experimentation or improvement to SLHS.


\subsection{Modifications to the Vector Representation}

In our implementation of SLHS we chose to represent belief vectors as red-black trees in
order to avoid storing the entire frame of discernment in memory: elements of the frame
that have zero belief mass assigned to them are simply not stored in the tree. While this
representation has some nice theoretical properties, such as the ability to map functions
across the vector in $O (n)$ time, and the ability to determine whether an element is or is
not a focal element in $O (\log n)$ time, we believe that improvements in the actual runtime
of the library may be achieved by switching to using a contiguous array.


\subsection{Implementing Memoization}

We have shown how some of the operators of Subjective Logic scale with respect
to the cardinalities of the frames of discernment involved. As we deal with
larger and larger frames, computing the results of the individual operators
will become more and more time consuming. If a single sub-expression appears
many times throughout a more complex subjective logic expression, it would be
beneficial to re-use a previously computed value. Instead, currently we would waste
valuable time recomputing the output for the same expression over and over.

One technique to avoid this costly recomputation is \emph{memoization} [cite].
At every operator invocation, we compute the value and store it in a table. If at
any time we require the same expression to be computed, we first look answer up in
the table. In a sense we would use additional memory in order to save time.


\subsection{Exploiting Parallelization}

Many operators of Subjective Logic appear to be easily parallelizable,
as the new opinions are computed by combining together the belief
masses of individual elements of the reduced powerset without
depending on any other elements. Therefore, attempting to introduce
parallelism to the implementations of the operators should be as easy
as modifying the underlying \emph{SLExpr} monad to utilize one of the
many Haskell libraries for parallel and concurrent computing
[cite]. Then the operators can be rewritten to compute their results
in parallel without any modification to the external interface to the
library. While we did not address the issue of parallelism in this
thesis, it appears, at least to the authors, to be a useful area of
future research.



\end{document}
