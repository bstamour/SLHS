\documentclass[thesis.tex]{subfiles}

\begin{document}

\chapter{Conclusion}
\label{chap:conclusion}

In this chapter we will present our concluding remarks, as well as discuss possible avenues for
future improvements to the SLHS library.






\section{Conclusion}





\section{Future Work}

In this section we will discuss areas for future experimentation or improvement to SLHS.


\subsection{Modifications to the Vector Representation}

% Maybe using a sorted vector instead of a red black tree will be faster
% on real hardware.




\subsection{Implementing Memoization}

We have shown how some of the operators of Subjective Logic scale with respect
to the cardinalities of the frames of discernment involved. As we deal with
larger and larger frames, computing the results of the individual operators
will become more and more time consuming. If a single sub-expression appears
many times throughout a more complex subjective logic expression, it would be
beneficial to re-use a previously computed value. Instead, currently we would waste
valuable time recomputing the output for the same expression over and over.

One technique to avoid this costly recomputation is \emph{memoization} [cite].
At every operator invocation, we compute the value and store it in a table. If at
any time we require the same expression to be computed, we first look answer up in
the table. In a sense we would use additional memory in order to save time.



\subsection{Exploiting Parallelization}

Many operators of Subjective Logic appear to be easily parallelizable,
as the new opinions are computed by combining together the belief
masses of individual elements of the reduced powerset without
depending on any other elements. Therefore, attempting to introduce
parallelism to the implementations of the operators should be as easy
as modifying the underlying \emph{SLExpr} monad to utilize one of the
many Haskell libraries for parallel and concurrent computing
[cite]. Then the operators can be rewritten to compute their results
in parallel without any modification to the external interface to the
library. While we did not address the issue of parallelism in this
thesis, it appears, at least to the authors, to be a useful area of
future research.



\end{document}
