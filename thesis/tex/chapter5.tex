\documentclass[thesis.tex]{subfiles}

\begin{document}


\chapter{Results and Analysis}
\label{chap:results-and-analysis}

In this chapter we will analyze SLHS by proving that the combinators terminate for
all input expressions, analyze the complexity of the Subjective Logic operators, and
finally demonstrate the power of SLHS by showcasing some example applications.





\section{Proofs of Termination}

In this section we will prove termination of our combinators and operators. That is,
for each Subjective Logic expression of finite size, we will prove that our combinators
will eventually terminate.




\begin{theorem}
  For every equation $e$ that is representible in SLSH, $e$ terminates.
\end{theorem}

\begin{proof}
  By induction on the length of $e$.
\end{proof}






\section{Analysis of Complexity}

In this section we will analyze the computational complexity of the Subjective Logic
operators that are available in SLHS. The time taken for each operator to compute its
output will be analyzed with respect to the cardinality of the frames of discernment
associated with each input opinion. With this in mind, we will not analyze the
binomial operators, as binomial opinions always have a fixed cardinality, and thus
an analysis based on frames of varying size would be meaningless.





\begin{theorem}
  Hypernomial abduction has time complexity $O (n log n)$.
\end{theorem}

\begin{proof}
  Immediate.
\end{proof}


\begin{theorem}
  Multinomial fission has time complexity $O (n log n)$.
\end{theorem}

\begin{proof}
  Immediate.
\end{proof}


\begin{theorem}
  Multinomial multiplication has time complexity $O (m \times n)$.
\end{theorem}

\begin{proof}
  Immediate.
\end{proof}







\section{Leveraging the Type System}





\section{Example Applications}






\end{document}
