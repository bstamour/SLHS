%
% generic formatting stuff:
% 1. Chapter X, Section Y
% 2. Make sure to fix up Josang's name.
% 3. Get a bunch of references to applications of SL
% 4. References for chapter 2. There should be tons and tons.
%

%
% All mentions of combinator => operator.
%


\documentclass[oneside, 12pt]{book}

\usepackage[top=1in, bottom=1in, left=1.5in, right=1in]{geometry}
\usepackage{subfiles}
\usepackage[nottoc,numbib]{tocbibind}
\usepackage{listings}
\usepackage{setspace}
\usepackage{amsthm}
\usepackage{amssymb,amsmath}
\usepackage{url}
\usepackage{verbatim}
\usepackage{graphicx}
\usepackage{tabularx}






\usepackage{mathptmx}% http://ctan.org/pkg/mathptmx








\newtheorem{theorem}{Theorem}[section]
\newtheorem{lemma}[theorem]{Lemma}
\newtheorem{proposition}[theorem]{Proposition}
\newtheorem{corollary}[theorem]{Corollary}

\lstloadlanguages{Haskell}

\lstnewenvironment{code}
    {\lstset{}%
      \csname lst@SetFirstLabel\endcsname}
    {\csname lst@SaveFirstLabel\endcsname}
    \lstset{
      basicstyle={\scriptsize\ttfamily\singlespacing},
      flexiblecolumns=false,
      basewidth={0.5em,0.45em},
      literate={+}{{$+$}}1 {/}{{$/$}}1 {*}{{$*$}}1 {=}{{$=$}}1
               {>}{{$>$}}1 {<}{{$<$}}1 {\\}{{$\lambda$}}1
               {\\\\}{{\char`\\\char`\\}}1
               {->}{{$\rightarrow$}}2 {>=}{{$\geq$}}2 {<-}{{$\leftarrow$}}2
               {<=}{{$\leq$}}2 {=>}{{$\Rightarrow$}}2
               {\ .\ }{{$\circ$}}2
               {>>}{{>>}}2 {>>=}{{>>=}}2
               {|}{{$\mid$}}1
    }

\lstnewenvironment{spec}
    {\lstset{}%
      \csname lst@SetFirstLabel\endcsname}
    {\csname lst@SaveFirstLabel\endcsname}
    \lstset{
      basicstyle={\scriptsize\ttfamily\singlespacing},
      flexiblecolumns=false,
      basewidth={0.5em,0.45em},
      literate={+}{{$+$}}1 {/}{{$/$}}1 {*}{{$*$}}1 {=}{{$=$}}1
               {>}{{$>$}}1 {<}{{$<$}}1 {\\}{{$\lambda$}}1
               {\\\\}{{\char`\\\char`\\}}1
               {->}{{$\rightarrow$}}2 {>=}{{$\geq$}}2 {<-}{{$\leftarrow$}}2
               {<=}{{$\leq$}}2 {=>}{{$\Rightarrow$}}2
               {\ .\ }{{$\circ$}}2
               {>>}{{>>}}2 {>>=}{{>>=}}2
               {|}{{$\mid$}}1
    }

\usepackage{setspace}

\newcommand{\uwinverytightsinglespacelen}{0.9}
\newcommand{\uwintightsinglespacelen}{1.0}
\newcommand{\uwinsinglespacelen}{1.1}
\newcommand{\uwinonehalfspacelen}{1.5}
\newcommand{\uwindoublespacelen}{2.0}
\newcommand{\uwinlistofspacelen}{1.5}
\newcommand{\uwindefaultspacelen}{\uwindoublespacelen}

\newcommand{\uwinverytightsinglespace}%
{\linespread{\uwinverytightsinglespacelen}}
\newcommand{\uwintightsinglespace}%
  {\linespread{\uwintightsinglespacelen}}
\newcommand{\uwinsinglespace}%
  {\linespread{\uwinsinglespacelen}}
\newcommand{\uwinonehalfspace}%
  {\linespread{\uwinonehalfspacelen}}
\newcommand{\uwindoublespace}%
  {\linespread{\uwindoublespacelen}}
\newcommand{\uwinlistofspace}%
  {\linespread{\uwinlistofspacelen}}
\newcommand{\uwindefaultspace}%
{\linespread{\uwindefaultspacelen}}

\newenvironment{uwinverytightsinglespaceenv}%
{\begin{spacing}{\uwinverytightsinglespacelen}}%
  {\end{spacing}}
\newenvironment{uwintightsinglespaceenv}%
  {\begin{spacing}{\uwintightsinglespacelen}}%
  {\end{spacing}}
\newenvironment{uwinsinglespaceenv}%
{\begin{spacing}{\uwinsinglespacelen}}%
  {\end{spacing}}
\newenvironment{uwinonehalfspaceenv}%
{\begin{spacing}{\uwinonehalfspacelen}}%
  {\end{spacing}}
\newenvironment{uwindoublespaceenv}%
{\begin{spacing}{\uwindoublespacelen}}%
  {\end{spacing}}
\newenvironment{uwinlistofspaceenv}%
{\begin{spacing}{\uwinlistofspacelen}}%
  {\end{spacing}}
\newenvironment{uwindefaultspaceenv}%
{\begin{spacing}{\uwindefaultspacelen}}%
  {\end{spacing}}






%\doublespacing
\long\def\ignore#1{}

\title{A Subjective Logic Library Constructed Using Monadic Higher Order Functions}

\author{Bryan St. Amour}

\begin{document}


\pagenumbering{roman}
%\setcounter{page}{2}


%---------------------------------------------------------------------------------------------------

\clearpage

%\currentpdfbookmark{Title Page}{titlepage}

\thispagestyle{empty}
\begin{center}
  \vspace*{1in}

  \begin{uwinonehalfspaceenv}
    \Large\textbf{A Subjective Logic Library Constructed Using Monadic Higher Order Functions}
  \end{uwinonehalfspaceenv}

  \vspace{\fill} %
  \begin{uwinonehalfspaceenv}
    By:\\*
    Bryan St. Amour
  \end{uwinonehalfspaceenv}
  \vspace{\fill}

  \normalsize
  A Thesis \\*
  Submitted to the Faculty of Graduate Studies \\*
  through the School of Computer Science \\*
  in Partial Fulfillment of the Requirements for \\*
  the degree of Master of Science at the \\*
  University of Windsor \\*

  \vspace{1in}
  Windsor, Ontario, Canada \\
  \vspace{0.5cm}
  2014 \\
  \vspace{0.5cm}
  \textcopyright \  2014 Bryan St. Amour
\end{center}

%---------------------------------------------------------------------------------------------------

\clearpage
\thispagestyle{empty}

\vspace*{\fill}

%\currentpdfbookmark{Copyright}{copyrightpage}%
\noindent \textcopyright{} 2014, Bryan St. Amour

\vspace{2ex}

\noindent All Rights Reserved. Absolutely no part of this document may
be reproduced, stored in a retrieval system, translated, in any form
or by any means electronic, mechanical, facsimile, photocopying, or
otherwise, without the prior written permission of the copyright
holder.

\vspace*{\fill}


%---------------------------------------------------------------------------------------------


% Ability to do centered horizontal rules for approval page...
\newcommand{\centeringrule}[2]{%
  \hspace{\fill}%
  \rule{#1}{#2}%
  \hspace{\fill}%
}

%----------------------------------------------------------------------------
% Thesis Approval Page
%   arg1 = thesis title
%   arg2 = thesis author
%   arg3 = defense date
%----------------------------------------------------------------------------
\newcommand{\uwinthesisapprovalpage}[3]{%
%  \cleartonextpage
  \clearpage
%  \currentpdfbookmark{Approved By}{approvedby}
  \thispagestyle{empty}

  \begin{center}
    \begin{uwinonehalfspaceenv}
      \Large\textbf{#1}
    \end{uwinonehalfspaceenv}

    \vspace{0.5cm}
    By:\\
    #2

    \vspace{0.5cm}

    \normalsize
    APPROVED BY: \\

    \begin{uwinverytightsinglespaceenv}
      \small

      \vspace{1in}
%      \vspace{0.335in}
      \centeringrule{3.5in}{0.5pt}
      \newline
      Dr. Richard Caron \\*
      Department of Mathematics and Statistics \\

      \vspace{1in}
%      \vspace{0.335in}
      \centeringrule{3.5in}{0.5pt}
      \newline
      Dr. Richard A. Frost \\*
      School of Computer Science \\

      \vspace{1in}
%      \vspace{0.335in}
      \centeringrule{3.5in}{0.5pt}
      \newline
      Dr. Robert D. Kent, Advisor \\*
      School of Computer Science \\

      \vspace{1in}
      \vspace{0.335in}
%      \centeringrule{3.5in}{0.5pt}
%      \newline
%      Dr. Scott Goodwin, Chair \\*
%      School of Computer Science \\[3ex]

      \hspace{\stretch{1}} #3
    \end{uwinverytightsinglespaceenv}
    \vfill
  \end{center}
}
% \pagebreak

\uwinthesisapprovalpage
  {A Subjective Logic Library Constructed Using Monadic Higher Order Functions}
  {Bryan St. Amour}
  {September 16, 2014}


%---------------------------------------------------------------------------------------------












\clearpage
% \phantomsection
\addcontentsline{toc}{chapter}{Author's Declaration of Originality}
\chapter*{Author's Declaration of Originality\markboth{\MakeUppercase{Author's Declaration of Originality}}{}}

\begin{uwindefaultspaceenv}
  I hereby certify that I am the sole author of this thesis and that
  no part of this thesis has been published or submitted for
  publication.

  I certify that, to the best of my knowledge, my thesis does not
  infringe upon anyone's copyright nor violate any proprietary rights
  and that any ideas, techniques, quotations, or any other material
  from the work of other people included in my thesis, published or
  otherwise, are fully acknowledged in accordance with the standard
  referencing practices. Furthermore, to the extent that I have
  included copyrighted material that surpasses the bounds of fair
  dealing within the meaning of the Canada Copyright Act, I certify
  that I have obtained a written permission from the copyright
  owner(s) to include such material(s) in my thesis and have included
  copies of such copyright clearances to my appendix.

  I declare that this is a true copy of my thesis, including any final
  revisions, as approved by my thesis committee and the Graduate
  Studies office, and that this thesis has not been submitted for a
  higher degree to any other University or Institution.
\end{uwindefaultspaceenv}


%---------------------------------------------------------------------------------------------


\clearpage
\chapter*{Abstract\markboth{\MakeUppercase{Abstract}}{}}
\addcontentsline{toc}{chapter}{Abstract}

%
% Maybe just rebuild this from scratch...
%

\begin{uwindoublespaceenv}
Subjective Logic is a recently emergent probabilistic logic system
that allows for reasoning under uncertainty. The fundamental
quantities include opinions based on ownership, beliefs, uncertainty,
and base rates, and a range of operators including logic, fusion and
reasoning over combinations of opinions. Though algebraically
expressive, there is a lack of software tooling to support
computation, such as code libraries, calculators, work benches and, in
particular, software for the development of decision support systems. With this
motivation, we present a complete design for a library of opinion data
structures and operators constructed from higher order functions that
are capable of representing and evaluating arbitrary well-formed
expressions of Subjective Logic.  By leveraging monads, mathematical
objects from Category Theory, we have enabled our operators to
propagate run-time errors from deeply nested sub-expressions and
control the access and updating of state, without sacrificing
compositionality. Furthermore, we have conducted a termination
analysis on the expression evaluator, and a complexity analysis on a
representative subset of the operators. We have also proposed and implemented
extensions to the set of Subjective Logic operators. We have implemented our
library as Subjective Logic in Haskell (SLHS) and provide examples of
how to create and compute the values of Subjective Logic expressions.
\end{uwindoublespaceenv}


%---------------------------------------------------------------------------------------------


\clearpage
\chapter*{\centering Dedication\markboth{\MakeUppercase{Dedication}}{}}
\addcontentsline{toc}{chapter}{Dedication}
\begin{center}
  \normalsize
  \textit{This thesis is dedicated to my late grandfather, Arthur
    Rigo. Though I wonder whether you would understand the content of
    this thesis, I've no doubt you'd be damn proud of me. This thesis
    concerns itself with automated reasoning systems, but there is
    just no reasoning with cancer. Here's to you, kemosabe.}
\end{center}


%---------------------------------------------------------------------------------------------


\clearpage
\chapter*{Acknowledgements\markboth{\MakeUppercase{Acknowledgements}}{}}
\addcontentsline{toc}{chapter}{Acknowledgements}

\begin{uwindefaultspaceenv}
No worthwhile academic endevour can be done in isolation; the age of the lone genius working
away in his tower are long behind us. Conducting academic research is a team effort, and this
thesis would not have come together without the assistance of some key players. I would first like
to thank my lab mates Paul Preney, Dave MacMillan, and Jeffery Drake for being there to bounce ideas
off of. Especially to Paul: we may have a friendly rivalry when it comes to meta-programming, but you
are by far the better programmer.

I would also like to extend my sincerest thanks to my thesis readers: Dr. Rick Caron from Mathematics and
Statistics, and Dr. Richard Frost from Computer Science. Your comments and suggestions have helped shape
my research in a positive way. Many thanks also go out to my thesis advisor Dr. Robert D. Kent. Bob, thank
you for everything you have done over the years, from the excellent conversations on everything from women
to quantum tunnelling, to the opportunities that you gave me for developing myself both academically and
industrially. Finally, thanks for believing in me even during the times that I had all but given up on myself.

I also wish to acknowledge the love and support of my amazing parents, Gary and Lee Ann St. Amour, and to my awesome
sister Kristen. Thank you for
always being there for me. I would also like to extend my endless gratitude to my wife, Chelsey St. Amour. I
started the Master's program as your boyfriend, and I'm finishing it as your husband. Words cannot express how
thankful I am of your love and support throughout this adventure.

Finally, I would like to thank the Canadian Institutes of Health Research (CIHR) and the Auto21 Network of Centres of Excellence for financial support.
\end{uwindefaultspaceenv}


%---------------------------------------------------------------------------------------------


\begin{uwindefaultspaceenv}

\tableofcontents

\listoffigures
\clearpage

\listoftables
\clearpage

\pagenumbering{arabic}

\subfile{chapter1}
\subfile{chapter2}
\subfile{chapter3}
\subfile{chapter4}
\subfile{chapter5}
\subfile{chapter6}


%\nocite{*} % Just for now...
\bibliography{bibliography}
\bibliographystyle{plain}


\chapter*{Vita Auctoris}
\addcontentsline{toc}{chapter}{Vita Auctoris}

Bryan St. Amour was born in 1987 and raised in Windsor, Ontario, Canada. He completed his undergraduate degree in
Computer Science from the University of Windsor in 2010, and his Master's degree in Computer Science from the same
institute in 2014.

%Bryan St. Amour was born and raised in Windsor, Ontario, Canada, and graduated from F.J. Brennan
%Catholic High School in 2005. Bryan completed his undergraduate degree in Computer Science from
%the University of Windsor in 2010, and expects to complete his Master's degree in Computer Science
%from the same institute in Fall 2014. He currently resides in Tecumseh, Ontario with his wife
%Chelsey St. Amour, and their four guinea pigs: Squeakers, Beauty, Scarlet, and Pumpkin.

\end{uwindefaultspaceenv}

\end{document}
