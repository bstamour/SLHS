\documentclass[thesis.tex]{subfiles}

\begin{document}


\chapter{SLHS: Subjective Logic in Haskell}
\label{chap:sl-in-haskell}


\section{Overview}

In this chapter we introduce our combinator library \emph{SLHS:
  Subjective Logic in Haskell}. SLHS is a monadic combinator library
for constructing and evaluating expressions of Subjective Logic. It
can be embedded into any existing Haskell project, and, through
Haskell's \emph{Foreign Function Interface}, can be utilized by other
programming languages, most notably $C$ and $C++$.

SLHS is designed to be simple to use: all Subjective Logic operators
take in \emph{expressions} as input, and return expressions as output.
An expression is simply a function, mapping some data (frames of
discernment, belief mass assignments, configuration information) to
some value - typically an opinion, but any value of any type, whether
it be rational numbers, strings, or some complex user-defined type.
It will be shown that these SL expressions, or \emph{SLExpr}s are a
kind of \emph{monad}, and thus can be used and combined using
pre-existing monad operators, or Haskell's convenient \emph{do-notation}
syntactic sugar.

We use the monad (and applicative and functor) operators liberally
within the implementation of SLHS, mostly to keep the code concise.


\section{Core Components}

In this section we will introduce components of the library that
are for internal use only. These include the implementation details
for objects such as the \emph{frame of discernment}, \emph{belief
  vectors}, as well as the the type \emph{SLExpr}, which is a monad.
All operators are defined to take and return objects of type
SLExpr, and the monadic interface controls the combination of
sub-expressions.

\subfile{./SLHS/Vector.lhs}
\subfile{./SLHS/Frame.lhs}
\subfile{./SLHS/Types.lhs}

\subfile{./SLHS/Opinions.lhs}
\subfile{./SLHS/Operators.lhs}
\subfile{./SLHS/Extensions.lhs}


\end{document}
