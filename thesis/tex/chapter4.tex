\documentclass[thesis.tex]{subfiles}

\begin{document}


\chapter{SLHS: Subjective Logic in Haskell}
\label{chap:sl-in-haskell}


%\section{Overview}

In this chapter we introduce our combinator library \emph{SLHS:
  Subjective Logic in Haskell}. SLHS is a monadic combinator library
for constructing and evaluating expressions of Subjective Logic. It
can be embedded into any existing Haskell project, and, through
Haskell's \emph{Foreign Function Interface}, can be utilized by other
programming languages, most notably $C$ and $C++$.

SLHS is designed to be simple to use: all Subjective Logic operators
take in \emph{expressions} as input, and return expressions as output.
An expression is simply a function, mapping some data (frames of
discernment, belief mass assignments, configuration information) to
some value - typically an opinion, but any value of any type, whether
it be rational numbers, strings, or some complex user-defined type.
It will be shown that these SL expressions, or \emph{SLExpr}s are a
kind of \emph{monad}, and thus can be used and combined using
pre-existing monad operators, or Haskell's convenient \emph{do-notation}
syntactic sugar.

We use the monad (and applicative and functor) operators liberally
within the implementation of SLHS, mostly to keep the code concise.


\section{Core Components}

In this section we will introduce components of the library that
are for internal use only. These include the implementation details
for objects such as the \emph{frame of discernment}, \emph{belief
  vectors}, as well as the the type \emph{SLExpr}, which is a monad.
All operators are defined to take and return objects of type
SLExpr, and the monadic interface controls the combination of
sub-expressions.

\subfile{./SLHS/Vector.lhs}
\subfile{./SLHS/Frame.lhs}
\subfile{./SLHS/Types.lhs}

\subfile{./SLHS/Opinions.lhs}
\subfile{./SLHS/Operators.lhs}
\subfile{./SLHS/Extensions.lhs}




\section{Limitations}
\label{sec:limitations}

While SLHS is a robust implementation of the opinions and operators of Subjective Logic,
our decision to represent all numbers as arbitrary-precision rational numbers imposes a
fundamental restriction on the kinds of data that the library can handle. Any computation
that involves the assignment of irrational numbers as belief masses cannot be represented
directly in our system. However, it is possible to modify SLHS to be able to handle such
values: one simply needs to either change the belief vectors to use values of type
\emph{Double} instead of \emph{Rational}, or better yet, represent the numeric type as
an additional type parameter to the belief vector. The latter would allow the user to
use any numerical type of his or her choosing.




%
% Under extensions of SL, mention how SL is incomplete as far as total operators. Operators
% seemingly added to fill some need... Perhaps discuss some information theory/entropy here...
%



\section{Summary}

In this chapter we introduced SLHS: Subjective Logic in Haskell, a combinator library
for representing and evaluating Subjective Logic expressions. We discussed the core
components of the library including the monads that represent the expressions, the
battery of Subjective Logic opinions and operators, and we concluded with a new operator
that is unique to the library. In the next chapter we will present a termination
analysis of the library, analyze the complexity of a handfull of the operators, discuss
how we leveraged the strong type system to catch errors at compile time, and demonstrate
the validity of the library by recomputing examples from the literature.







\end{document}
