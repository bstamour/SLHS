\documentclass[thesis.tex]{subfiles}

\begin{document}


\chapter{SLHS: Subjective Logic in Haskell}


\section{Overview}

In this chapter we introduce our combinator library \emph{SLHS:
  Subjective Logic in Haskell}. SLHS is a monadic combinator library
for constructing and evaluating expressions of Subjective Logic. It
can be embedded into any existing Haskell project, and, through
Haskell's \emph{Foreign Function Interface}, can be utilized by other
programming languages, most notably $C$ and $C++$.


\section{Core Components}

In this section we will introduce components of the library that
are for internal use only. These include the implementation details
for objects such as the \emph{frame of discernment}, \emph{belief
  vectors}, as well as the the type \emph{SLExpr}, which is a monad.
All operators are defined to take and return objects of type
SLExpr, and the monadic interface controls the combination of
sub-expressions.

\subfile{./SLHS/Vector.lhs}
\subfile{./SLHS/Frame.lhs}
\subfile{./SLHS/Types.lhs}


\section{Opinions}

\subfile{./SLHS/Opinions.lhs}


\section{Operators}

In this section we will discuss the implementation details of the
Subjective Logic operators that are provided by SLHS. The following
notation is used for the operators:

\begin{itemize}
  \item We denote binary operators with a trailing exclamation mark
    $!$ in order to avoid conflicting with Haskell's mathematical
    operators. For example, binomial addition is denoted as $+!$.
  \item We use tildas as a prefix to denote $co-$ operations. For
    example, the binomial co-multiplication operator is denoted as
    $\sim *!$.
  \item All n-ary operators, where $n > 2$ are denoted as simple
    functions, instead of symbolic operators.
\end{itemize}

Furthermore, every operator is presented in it's most general
form. For example, instead of presenting two operators for
\emph{averaging fusion} (one for multinomial opinions, and another for
hyper opinions) we implement only the version for hyper opinions.

\subfile{./SLHS/Operators/Binomial.lhs}
\subfile{./SLHS/Operators/Multinomial.lhs}
\subfile{./SLHS/Operators/Hyper.lhs}

\end{document}
