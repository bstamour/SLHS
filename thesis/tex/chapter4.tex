
\documentclass[thesis.tex]{subfiles}

\begin{document}

\chapter{SLHS: Subjective Logic in Haskell}

\section{Overview}

In this chapter we introduce our combinator library \emph{SLHS: Subjective Logic
in Haskell}. SLHS is a monadic combinator library for constructing and evaluating
expressions of Subjective Logic. It can be embedded into any existing Haskell
project, and, through Haskell's \emph{Foreign Function Interface}, can be utilized
by other programming languages, most notably $C$ and $C++$.

%The following sections are \emph{literate Haskell}, meaning that the same
%\LaTeX source used to generate this thesis document can be utilized as a standalone
%Haskell module.



%\subfile{./SLHS/Core.lhs}
%\subfile{./SLHS/Base.lhs}





\subfile{./SLHS/Types.lhs}
\subfile{./SLHS/Opinions.lhs}




\section{Operators}

In this section we will discuss the implementation details of the Subjective Logic
operators that are provided by SLHS. The following notation is used for the operators:

\begin{itemize}
  \item We denote binary operators with a trailing exclamation mark $!$ in order to
    avoid conflicting with Haskell's mathematical operators. For example, binomial addition
    is denoted as $+!$.
  \item We use tildas as a prefix to denote $co-$ operations. For example, the binomial
    co-multiplication operator is denoted as $\sim *!$.
  \item All n-ary operators, where $n > 2$ are denoted as simple functions, instead of
    symbolic operators.
\end{itemize}

Furthermore, every operator is presented in it's most general form. For example, instead of
presenting two operators for \emph{averaging fusion} (one for multinomial opinions, and another
for hyper opinions) we implement only the version for hyper opinions.





\subfile{./SLHS/Operators/Binomial.lhs}
\subfile{./SLHS/Operators/Multinomial.lhs}
\subfile{./SLHS/Operators/Hyper.lhs}

\end{document}
